\documentclass[a4j,papersize,disablejfam,slide,20pt]{jsarticle}
\usepackage{graphicx,xcolor}
\usepackage{lastpage}
\usepackage{fancyhdr}
\renewcommand{\headrulewidth}{0.0pt}
\pagestyle{fancy}
\lhead{}
\chead{}
\rhead{}
\lfoot{}
\cfoot{\thepage{}/{}\pageref{LastPage}}
\rfoot{}
\usepackage[T1]{fontenc}
\usepackage{textcomp}
\usepackage[utf8]{inputenc}
\usepackage{bm}
\usepackage{comment}

\begin{comment}
 documentclass から17行はコピペ.意味はわからず.
 参考:
 	「何かを書き留める何か LaTeX + jsarticle + slide でスライドを作る」
 	url: http://xaro.hatenablog.jp/entry/2013/09/26/004920
\end{comment}

\usepackage{amsmath}
\usepackage{ascmac}
\allowdisplaybreaks[1]
\newtheorem{prop}{定理}
\newtheorem{proof}{証明}
\def\qed{\hfill $\Box$} %証明終了
\def\vector#1{\mbox{\boldmath $#1$}} %ベクトルを太字表示
\def\norm#1{\mbox{$\left\| #1 \right\|$}} %ノルム
\def\det#1{\mbox{${\rm det} \left( #1 \right)$}} %行列式
\def\diag#1{\mbox{${\rm diag} \left( #1 \right)$}} %行列の対角成分
\def\Exp#1{\mbox{${\rm E} \left[ #1 \right]$}} %期待値
\def\Var#1{\mbox{${\rm V} \left[ #1 \right]$}} %分散
\def\Cov#1#2{\mbox{${\rm Cov} \left[ #1,\ #2 \right]$}} %共分散
\def\exp#1{\mbox{${\rm exp} \left( #1 \right)$}} %指数関数

\begin{document}

\title{ ゼミ資料\\待ち行列理論と板の動きへの応用}
\author{学籍番号:201311324\\百合川尚学}
\maketitle

\section{待ち行列理論の導入}
	興味があること
	\begin{itemize}
		\item 観測を始めて$t$時間経過した後のシステム内の客数.
    	\item システムにいる客数が初期状態から$0$になるまでの時間の分布.
    	\item システムを最良気配に見立てると,最良気配にかかる注文の数量の変化の分布
    	を考えることになる.
	\end{itemize}
	\begin{picture}(100,100)
    	\put( 55, 55){\framebox(20,20){客}}
	\end{picture}

\newpage
\section{基礎理論まとめ}
	参考文献:
    \begin{enumerate}
    	\item Suzuki, Queueing, Shokabo, 1972, pp. 20-65.
        \item Endo, Zuo, Kishimoto, 
        Modelling Intra-day Stock Price Changes In Terms of
        a Continuous Double Auction System, 
        The Japan Society for Industrial and Applied Mathematics, 
        Vol.16 , No.3, 2006, pp.305-316.
        \item Li, Hui, Endo, Kishimoto, A Quantitative Model for Intraday Stock Price
         Changes Based on Order Flows, 
         J Syst Sci Complex, 2014, 27: 208-224.
    \end{enumerate}
    上記文献$2$と$3$に従い,板は最良気配のみを考え,板が動くことは最良気配値が動くこととする.\\\\
    注文の種類:上記文献$2$と$3$に従い,次の4種類のみを考える.
    \begin{itemize}
    	\item 指値買い/売り注文 (最良買い/売り気配の数量を増加する.)
        \item 成行買い/売り注文 (最良買い/売り気配の数量を減少する.)
    \end{itemize}
    \begin{picture}(100,100)
    	
	\end{picture}
    
\newpage
\section{基礎理論まとめ}
    \begin{screen}
    或るシステムがあり,そのシステムには或る確率分布に従った時間間隔で客が訪れ,
    或る確率分布に従った時間だけサービスを受け退場する.到着の時間間隔およびサービス時間
    は客ごとに独立であると考える.
    \end{screen}
    到着時間の分布について,\\
    観測開始時刻を$T_0$,始めの客が到着する時刻を$T_1$,$2$番目の客が到着する時刻を$T_2$,
    $\cdots$ ,として系列$\{T_n\}_{n=0}^{\infty}$を得る.各時間間隔$T_n - T_{n-1}, n=0,1,2,\cdots$
    はどの二つも互いに独立で同一な確率分布(到着分布)に従う.\\
    \begin{description}
    	\item[到着分布の例:$k-$アーラン分布\ $(k-Erlang\ distribution)$]\mbox{}\\
    		分布関数を$E_k(x),\ -\infty < x < \infty$と表すと,
    		\begin{equation}
    			E_k(x) \equiv
        		\begin{cases}
        			1 - \exp{-\lambda k x} \left( 1 + \frac{\lambda k x}{1!} + \cdots + \frac{(\lambda k x)^{k-1}}{(k-1)!} \right) & \text{$x \geq 0$}\\
    				0 & \text{$x < 0$}
        		\end{cases}
    		\end{equation}
            平均 $\frac{1}{\lambda}$,分散 $\frac{1}{k\lambda^2}$,
            特性関数 $\phi_{E_k}(t) = \left( 1 - \frac{it}{\lambda k} \right)^{-k}$($i$:虚数単位).(付録$1$参照)
    \end{description}
    
%付録
\scriptsize
\newpage
\section{付録1}
	\begin{description}
    	\item[到着分布の例:$k-$アーラン分布\ $(k-Erlang\ distribution)$]\mbox{}\\
    		\begin{equation}
    			E_k(x) \equiv
        		\begin{cases}
        			1 - \exp{-\lambda k x} \left( 1 + \frac{\lambda k x}{1!} + \cdots + \frac{(\lambda k x)^{k-1}}{(k-1)!} \right) & \text{$x \geq 0$}\\
    				0 & \text{$x < 0$}
        		\end{cases}
    		\end{equation}
        平均,分散,特性関数を計算する.
        密度関数
        \begin{eqnarray}
            f(x) &=& E_k'(x) \\&=& 
            \begin{cases}
        			\lambda k \exp{-\lambda k x} \left( 1 + \frac{\lambda k x}{1!} + \cdots + \frac{(\lambda k x)^{k-1}}{(k-1)!} \right)
                    - \lambda k \exp{-\lambda k x} \left( 1 + \frac{\lambda k x}{1!} + \cdots + \frac{(\lambda k x)^{k-2}}{(k-2)!} \right) & \text{$x \geq 0$}\\
    				0 & \text{$x < 0$}
        	\end{cases} 
            \\&=&
            \begin{cases}
        			\lambda k \exp{-\lambda k x} \frac{(\lambda k x)^{k-1}}{(k-1)!} & \text{$x \geq 0$}\\
    				0 & \text{$x < 0$}
        	\end{cases}.
        \end{eqnarray}
        これは$Gamma$分布\ $G_A(k, \frac{1}{\lambda k})$の密度関数である.従って一般の
        $Gamma$分布\ $G_A(\alpha, \beta)$について平均,分散,特性関数を計算する方が楽である.\\
        特性関数 : 確率変数 $X \sim G_A(\alpha, \beta)$ について,
        \begin{eqnarray}
			\phi(t) &=& E[e^{itX}] \\
			&=& \int_{0}^{\infty} e^{itx} \frac{1}{\Gamma(\alpha)\beta^\alpha} x^{\alpha-1} e^{-\frac{x}{\beta}} dx \\
			&=& \int_{0}^{\infty} \frac{1}{\Gamma(\alpha)\beta^\alpha} x^{\alpha-1} e^{-(\frac{1}{\beta}-it)x} dx	\\
			&=& \lim_{R \to \infty} \int_{0}^{R} \frac{1}{\Gamma(\alpha)\beta^\alpha} x^{\alpha-1} e^{-(\frac{1}{\beta}-it)x} dx \\
			&=& \lim_{R \to \infty} \frac{1}{\Gamma(\alpha)\beta^\alpha} (\frac{\beta}{1-i \beta t})^\alpha \int_{0}^{\frac{R}{\beta}-itR} z^{\alpha-1} e^{-z} dz.
		\end{eqnarray}
        ここで
		\[
			\int_{0}^{\frac{R}{\beta}-itR} z^{\alpha-1} e^{-z} dz
		\]
		について複素積分を考える.
        積分路を$\Gamma \equiv \Gamma_1 \cup \Gamma_2 \cup \Gamma_3$として,被積分関数が$\mathbb{C}$の整関数であることから$\Gamma$および内部領域に孤立特異点は存在しない.
		積分の向きは左回りとして,$Cauchy$の積分定理が成り立つので
		\[
			\oint_{\Gamma} z^{\alpha-1} e^{-z} dz = 0
		\]
		が成り立つ.
		$\Gamma_2$上の積分は
		\begin{eqnarray}
			\left|\int_{\Gamma_2} z^{\alpha-1} e^{-z} dz\right| 
			&=& \left|\int_{-tR}^{0} (\frac{R}{\beta}+iy})^{\alpha-1} e^{-\frac{R}{\beta}-iy} i dy\right| \\
			&\leq& \int_{-tR}^{0} (\frac{R}{\beta}+|y|})^{\alpha-1} e^{-\frac{R}{\beta}} dy.
		\end{eqnarray}
		任意の$\epsilon > 0$に対し$t$について定まる或る$R_1(t)$が存在して,$R > R_1(t)$ならば
		\[
			\int_{-tR}^{0} (\frac{R}{\beta}+|y|})^{\alpha-1} e^{-\frac{R}{\beta}} dy < \epsilon
		\]
		が成り立つ.$\Gamma_3$上の積分は
		\[
			\int_{\frac{R}{\beta}}^{0} z^{\alpha-1} e^{-z} dz = -\int_{0}^{\frac{R}{\beta}} z^{\alpha-1} e^{-z} dz.
		\]
		これも広義積分は収束するので,任意の$\epsilon > 0$に対し或る$R_2$が存在して,$R > R_2$ならば
		\[
			\Gamma(\alpha)-\epsilon < \int_{0}^{\lambda R} z^{\alpha-1} e^{-z} dx \leq \Gamma(\alpha).
		\]
		従って,$R > max\{R_1(t), R_2\}$と置いて
		\begin{eqnarray}
			\left|\int_{\Gamma_1} z^{\alpha-1} e^{-z} dz -  \Gamma(\alpha)\right|
			&=& \left|-\int_{\Gamma_2} z^{\alpha-1} e^{-z} dz
		    	   -\int_{\Gamma_3} z^{\alpha-1} e^{-z} dz - \Gamma(\alpha)\right| < 2 \epsilon.
		\end{eqnarray}
		$\epsilon$は任意であるから
		\[
			\lim_{R \to \infty} \frac{1}{\Gamma(\alpha)\beta^\alpha} (\frac{\beta}{1-i \beta t})^\alpha \int_{0}^{\frac{R}{\beta}-itR} z^{\alpha-1} e^{-z} dz 
			= (\frac{1}{1-i \beta t})^\alpha
		\]
		が成り立つ.$t \leq 0$の場合も同じ結論となる.
        $\qed$
    \end{description}
    
\newpage
\section{付録2}
\end{document}
